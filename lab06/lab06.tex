%!TEX program = xelatex
\documentclass[dvipsnames, svgnames,a4paper,11pt]{article}
% ----------------------------------------------------
%   吉林大学通信工程学院信号与系统实验报告
%   原作者:Huanyu Shi,2019级
%     知乎:https://www.zhihu.com/people/za-ran-zhu-fu-liu-xing
%     Github:https://github.com/huanyushi/SYSU-SPA-Labreport-Template
%   在原基础上魔改了一些,更加贴近吉林大学实验格式。
% ----------------------------------------------------

\input{Settings} % 导入模板的相关设置
\usepackage{lipsum}
\usepackage{amsmath}


%---------------------------------------------------------------------
%	正文
%---------------------------------------------------------------------

\begin{document}

\begin{table}
  \raggedleft
	\renewcommand\arraystretch{1.7}
	\begin{tabular}{|c|p{4em}|}
	\hline
	成绩 &  \\
	\hline
	教师签字 &   \\
	\hline
	\end{tabular}
\end{table}

\begin{center}
	{\kaishu \LARGE   \quad  \quad 通  \quad 信  \quad 工  \quad 程  \quad 学  \quad 院 }
  \newline
  \newline
  \newline
  \newline
  \newline
  {\kaishu \Huge 实 \quad  \quad  \quad 验  \quad  \quad  \quad 报 \quad  \quad  \quad 告}
  \newline
  \newline
  \newline
  \newline
  \newline
  {\songti \Huge  ( \quad  信  \quad 号  \quad 与  \quad 系  \quad 统 \quad)}
  \newline
  \newline
  \newline
  \newline
  \newline
  {\songti  \LARGE 实验题目:非正弦周期信号的分解与合成  \quad  \quad \quad}
\end{center}



\begin{table}[b]
	\renewcommand\arraystretch{1.7}
	\begin{tabularx}{\textwidth}{|X|X|X|X|}
	\hline
	专业:& 通信工程 &年级:& 2022级\\
	\hline
	姓名:& 苏睿杰  & 学号:& 20220826\\
	\hline
	实验时间:& 2023年11月24日 & 班级:& 42 \\
	\hline
	\end{tabularx}
\end{table}


%\clearpage
%\tableofcontents

\clearpage
\setcounter{section}{0}
\section{实验十六 \quad 非正弦周期信号的分解与合成}
\subsection*{一、实验目的}
\begin{enumerate}
  \item 用实验方法观测非正弦周期信号的分解,并与其傅利叶级数各项的频率与系数作比较。 
  \item 观测基波和其谐波的合成。
\end{enumerate}

\subsection*{二、仪器设备}
\begin{enumerate}
  \item 实验箱一台。
  \item 数字示波器。
\end{enumerate}

\subsection*{三、原理说明}
任何信号都是由各种不同频率、幅度和初相的正弦波叠加而成的。由非正弦周期信号傅里叶级数展开式可知,各次谐波为基波频率的整数倍。而第一个非周期信号包含了从零到无穷大的所有频率成分,其幅度将随谐波次数的增高而减小,直至无穷小,将被测信号加到分别调谐于其基波和各次谐波频率的电路上。从每一带通滤器的输出端可以用示波器观察到相应频率的正弦波。本实验所用的被测信号是选用 $50Hz$ 的方波、矩形波、三角波、全波和半波等。

\subsection*{四、理论值的计算}
\begin{figure}[htbp]
  \centering
  \begin{minipage}[t]{0.48\textwidth}
  \centering
  \includegraphics[width=9cm]{1.png}
  \caption{全波整流信号}
  \end{minipage}
  \begin{minipage}[t]{0.48\textwidth}
  \centering
  \includegraphics[width=9cm]{2.png}
  \caption{半波整流信号}
  \end{minipage}
\end{figure}

\begin{enumerate}
  \item 基频为50Hz的方波的谐波型傅里叶级数
    
    将方波信号的傅里叶级数展开:

    \begin{equation*}
      f(t) = a_0 + \sum_{n = 1}^{\infty} \left [ a_n \cos (n\omega t) + b_n \sin (n\omega t) \right ]
    \end{equation*}

    利用公式可算得

    \begin{align*}
      & a_0 = \dfrac{1}{T}\int_{0}^{T}f(t)dt = 0 \\
      & a_n = \dfrac{2}{T}\int_{0}^{T}f(t)\cos (n\omega t)dt = 0 \\
      & b_n = \dfrac{2}{T}\int_{0}^{T}f(t)\sin (n\omega t)dt = \dfrac{4}{n\pi}\sin^2 (\dfrac{n\pi}{2})
    \end{align*}
      
    又由于

    \begin{align*}
      & A_n = \sqrt{a_n^2 + b_n^2} = \dfrac{4}{n\pi}\sin^2 (\dfrac{n\pi}{2})\\
      & \varphi_n = -\arctan (\dfrac{b_n}{a_n}) = -\dfrac{\pi}{2}
    \end{align*}
      
    最后可得到

    \begin{equation*}
      f(t) = \sum_{i = 1}^{\infty} \dfrac{4}{n\pi}\sin^2 (\dfrac{n\pi}{2}) \cos(n\omega t -\dfrac{\pi}{2})
    \end{equation*}

  \item 基频为50Hz的三角波的谐波型傅里叶级数
  
  将三角波信号的傅里叶级数展开:

  \begin{equation*}
    f(t) = a_0 + \sum_{n = 1}^{\infty} \left [ a_n \cos (n\omega t) + b_n \sin (n\omega t) \right ]
  \end{equation*}

  利用公式可算得

  \begin{align*}
    & a_0 = \dfrac{1}{T}\int_{0}^{T}f(t)dt = 0 \\
    & a_n = \dfrac{2}{T}\int_{0}^{T}f(t)\cos (n\omega t)dt = 0 \\
    & b_n = \dfrac{2}{T}\int_{0}^{T}f(t)\sin (n\omega t)dt = \dfrac{8}{(n\pi)^2}\sin (\dfrac{n\pi}{2})
  \end{align*}
    
  又由于

  \begin{align*}
    & A_n = \sqrt{a_n^2 + b_n^2} = \dfrac{8}{(n\pi)^2}\sin (\dfrac{n\pi}{2})\\
    & \varphi_n = -\arctan (\dfrac{b_n}{a_n}) = -\dfrac{\pi}{2}
  \end{align*}
    
  最后可得到

  \begin{equation*}
    f(t) = \sum_{i = 1}^{\infty} \dfrac{8}{(n\pi)^2}\sin (\dfrac{n\pi}{2})\cos(n\omega t - \dfrac{\pi}{2}) 
  \end{equation*}


  \item 基频为50Hz的锯齿波的谐波型傅里叶级数

  将锯齿波信号的傅里叶级数展开:

  \begin{equation*}
    f(t) = a_0 + \sum_{n = 1}^{\infty} \left [ a_n \cos (n\omega t) + b_n \sin (n\omega t) \right ]
  \end{equation*}

  利用公式可算得

  \begin{align*}
    & a_0 = \dfrac{1}{T}\int_{0}^{T}f(t)dt = \dfrac{1}{2} \\
    & a_n = \dfrac{2}{T}\int_{0}^{T}f(t)\cos (n\omega t)dt = \dfrac{2}{n^2\omega^2 T^2} [1 - \cos(n\omega T)] = 0 \\
    & b_n = \dfrac{2}{T}\int_{0}^{T}f(t)\sin (n\omega t)dt = \dfrac{1}{n\pi}
  \end{align*}
    
  又由于

  \begin{align*}
    & A_n = \sqrt{a_n^2 + b_n^2} = \dfrac{1}{n\pi}\\
    & \varphi_n = -\arctan (\dfrac{b_n}{a_n}) = -\dfrac{\pi}{2}
  \end{align*}
    
  最后可得到

  \begin{equation*}
    f(t) = \dfrac{1}{2} + \sum_{i = 1}^{\infty} \dfrac{1}{n\pi} \cos(n\omega t - \dfrac{\pi}{2})
  \end{equation*}



\end{enumerate}



\subsection*{五、实验内容及步骤}
\begin{enumerate}
  \item 调节函数信号发生器,使其输出为 $50Hz$ 方波。将其接至该实验模块的输入端,再细调函数信号发生器的输出,使 $50Hz$(基波)的 BPF 模块有最大的输出。然后,将各带通滤波器的输出分别接至示波器和交流亳伏表,观测各次谐波的频率和幅度,并记录之。
  \item 将方波分解所得的基波和三次谐波分量接至加法器的相应输入端,观测加法器的输出波形,并记录所得的波形。
  \item 再将方波的五次谐波分量加到加法器的相应输入端,观测相加后的波形,记录之。
  \item 分别输入 $50Hz$ 的矩形波、三角波、全波和半波信号,并分别观测谐波分量,记录波形的幅值及频率。
  \item 在加法器的输入端接入相应的各谐波分量,进行信号的合成实验,观察频率失真,并记录结果。
\end{enumerate}

\subsection*{六、数据记录及整理}
\begin{enumerate}
  \item 半波的分解和合成
    
    信号的分解
    \begin{figure}[htbp]
      \centering
      \begin{minipage}[t]{0.48\textwidth}
      \centering
      \includegraphics[width=6cm]{3.1.png}
      \caption{基波,频率为 49.2Hz,峰峰值为 6.12V}
      \end{minipage}
      \begin{minipage}[t]{0.48\textwidth}
      \centering
      \includegraphics[width=6cm]{3.2.png}
      \caption{二级谐波,频率为 110.2Hz,峰峰值为 9.88V}
      \end{minipage}

      \begin{minipage}[t]{0.48\textwidth}
      \centering
      \includegraphics[width=6cm]{3.3.png}
      \caption{三级谐波,频率为 149.8Hz,峰峰值为 6.60V}
      \end{minipage}
      \centering
      \begin{minipage}[t]{0.48\textwidth}
      \centering
      \includegraphics[width=6cm]{3.4.png}
      \caption{四级谐波,频率为 208.7Hz,峰峰值为 3.12V}
      \end{minipage}

      \begin{minipage}[t]{0.48\textwidth}
      \centering
      \includegraphics[width=6cm]{3.5.png}
      \caption{五级谐波,频率为 248Hz,峰峰值为 0.196V}
      \end{minipage}
      \begin{minipage}[t]{0.48\textwidth}
      \centering
      \includegraphics[width=6cm]{3.6.png}
      \caption{六级谐波,频率为 336Hz,峰峰值为 0.54V}
      \end{minipage}

      \begin{minipage}[t]{0.48\textwidth}
        \centering
        \includegraphics[width=5cm]{9.png}
        \caption{直流分量,值为-3V}
      \end{minipage}
    \end{figure}

    \newpage
    信号的合成
    \begin{figure}[htbp]
      \centering
      \begin{minipage}[t]{0.48\textwidth}
      \centering
      \includegraphics[width=7cm]{4.1.png}
      \caption{直流和基波合成,频率为 50.01Hz,峰峰值为 6.80V}
      \end{minipage}
      \begin{minipage}[t]{0.48\textwidth}
      \centering
      \includegraphics[width=7cm]{4.2.png}
      \caption{直流,一,二级谐波合成,频率为 50.03Hz,峰峰值为 6.62V}
      \end{minipage}

      \begin{minipage}[t]{0.48\textwidth}
      \centering
      \includegraphics[width=7cm]{4.3.png}
      \caption{直流,一,二,三级谐波,频率为 49.87Hz,峰峰值为 6.24V}
      \end{minipage}
      \centering
      \begin{minipage}[t]{0.48\textwidth}
      \centering
      \includegraphics[width=7cm]{4.4.png}
      \caption{直流,一,二,三,四级谐波,频率为 46.12Hz,峰峰值为 6.52V}
      \end{minipage}

      \begin{minipage}[t]{0.48\textwidth}
      \centering
      \includegraphics[width=7cm]{4.5.png}
      \caption{直流,一,二,三,四,五级谐波,频率为 49.99Hz,峰峰值为 6.31V}
      \end{minipage}
      \begin{minipage}[t]{0.48\textwidth}
      \centering
      \includegraphics[width=7cm]{4.6.png}
      \caption{直流,一,二,三,四,五,六级谐波,频率为 50.23Hz,峰峰值为 6.57V}
      \end{minipage}
    \end{figure}

  \newpage

  \item 全波的分解和合成
  
    信号的分解
    \begin{figure}[htbp]
      \centering
      \begin{minipage}[t]{0.48\textwidth}
      \centering
      \includegraphics[width=6cm]{3.1.png}
      \caption{一级谐波,频率为 100.8Hz,峰峰值为 0.176V}
      \end{minipage}
      \begin{minipage}[t]{0.48\textwidth}
      \centering
      \includegraphics[width=6cm]{3.2.png}
      \caption{二级谐波,频率为 199.2Hz,峰峰值为 5.24V}
      \end{minipage}

      \begin{minipage}[t]{0.48\textwidth}
      \centering
      \includegraphics[width=6cm]{3.3.png}
      \caption{三级谐波,频率为 291.1Hz,峰峰值为 0.56V}
      \end{minipage}
      \centering
      \begin{minipage}[t]{0.48\textwidth}
      \centering
      \includegraphics[width=6cm]{3.4.png}
      \caption{四级谐波,频率为 374.3Hz,峰峰值为 1.52V}
      \end{minipage}

      \begin{minipage}[t]{0.48\textwidth}
      \centering
      \includegraphics[width=6cm]{3.5.png}
      \caption{五级谐波,频率为 499.8Hz,峰峰值为 0.24V}
      \end{minipage}
      \begin{minipage}[t]{0.48\textwidth}
      \centering
      \includegraphics[width=6cm]{3.6.png}
      \caption{六级谐波,频率为 572.1Hz,峰峰值为 6.52V}
      \end{minipage}

      \begin{minipage}[t]{0.48\textwidth}
        \centering
        \includegraphics[width=5cm]{9.png}
        \caption{直流分量,值为 -0.68V}
      \end{minipage}
    \end{figure}

    \newpage
    信号的合成
    \begin{figure}[htbp]
      \centering
      \begin{minipage}[t]{0.48\textwidth}
      \centering
      \includegraphics[width=7cm]{5.1.png}
      \caption{直流和基波合成,频率为 100.01Hz,峰峰值为 5.80V}
      \end{minipage}
      \begin{minipage}[t]{0.48\textwidth}
      \centering
      \includegraphics[width=7cm]{5.2.png}
      \caption{直流,一,二级谐波合成,频率为 100.03Hz,峰峰值为 5.62V}
      \end{minipage}

      \begin{minipage}[t]{0.48\textwidth}
      \centering
      \includegraphics[width=7cm]{5.3.png}
      \caption{直流,一,二,三级谐波,频率为 99.87Hz,峰峰值为 6.24V}
      \end{minipage}
      \centering
      \begin{minipage}[t]{0.48\textwidth}
      \centering
      \includegraphics[width=7cm]{5.4.png}
      \caption{直流,一,二,三,四级谐波,频率为 96.12Hz,峰峰值为 6.52V}
      \end{minipage}

      \begin{minipage}[t]{0.48\textwidth}
      \centering
      \includegraphics[width=7cm]{5.5.png}
      \caption{直流,一,二,三,四,五级谐波,频率为 99.99Hz,峰峰值为 6.31V}
      \end{minipage}
      \begin{minipage}[t]{0.48\textwidth}
      \centering
      \includegraphics[width=7cm]{5.6.png}
      \caption{直流,一,二,三,四,五,六级谐波,频率为 100.23Hz,峰峰值为 6.57V}
      \end{minipage}
    \end{figure}



  \newpage
  \item 三角波的分解和合成
    
    信号的分解
    \begin{figure}[htbp]
      \centering
      \begin{minipage}[t]{0.48\textwidth}
      \centering
      \includegraphics[width=6cm]{3.1.png}
      \caption{基波,频率为 50.12Hz,峰峰值为 6.4V}
      \end{minipage}
      \begin{minipage}[t]{0.48\textwidth}
      \centering
      \includegraphics[width=6cm]{3.3.png}
      \caption{三级谐波,频率为 153.22Hz,峰峰值为 2.16V}
      \end{minipage}

      \begin{minipage}[t]{0.48\textwidth}
      \centering
      \includegraphics[width=6cm]{3.5.png}
      \caption{五级谐波,频率为 251.4Hz,峰峰值为 2.1V}
      \end{minipage}
      \begin{minipage}[t]{0.48\textwidth}
      \centering
      \includegraphics[width=6cm]{9.png}
      \caption{直流分量,值为 -1.06V}
      \end{minipage}
    \end{figure}

    \newpage
    信号的合成
    \begin{figure}[htbp]
      \centering
      \begin{minipage}[t]{0.48\textwidth}
      \centering
      \includegraphics[width=7cm]{6.1.png}
      \caption{直流和基波合成,频率为 49.12Hz,峰峰值为 6.5V}
      \end{minipage}
      
      \begin{minipage}[t]{0.48\textwidth}
      \centering
      \includegraphics[width=7cm]{6.2.png}
      \caption{直流和一,三级谐波合成,频率为 50.02Hz,峰峰值为 6.23V}
      \end{minipage}

      \begin{minipage}[t]{0.48\textwidth}
      \centering
      \includegraphics[width=7cm]{6.3.png}
      \caption{直流和一,三,五级谐波合成,频率为 50.01Hz,峰峰值为 6.48V}
      \end{minipage}
    \end{figure}
    

  \newpage
  \item 方波的分解和合成
    
    信号的分解
    \begin{figure}[htbp]
      \centering
      \begin{minipage}[t]{0.48\textwidth}
      \centering
      \includegraphics[width=6cm]{3.1.png}
      \caption{一级谐波,频率为 49.97Hz,峰峰值为 20V}
      \end{minipage}
      \begin{minipage}[t]{0.48\textwidth}
      \centering
      \includegraphics[width=6cm]{3.3.png}
      \caption{三级谐波,频率为 153.2Hz,峰峰值为 7.20V}
      \end{minipage}

      \begin{minipage}[t]{0.48\textwidth}
      \centering
      \includegraphics[width=6cm]{3.5.png}
      \caption{五级谐波,频率为 245.6Hz,峰峰值为 4.48V}
      \end{minipage}
      \begin{minipage}[t]{0.48\textwidth}
      \centering
      \includegraphics[width=6cm]{9.png}
      \caption{直流分量,值为 -0.6V}
      \end{minipage}
    \end{figure}

    \newpage
    信号的合成
    \begin{figure}[htbp]
      \centering
      \begin{minipage}[t]{0.48\textwidth}
      \centering
      \includegraphics[width=7cm]{7.1.png}
      \caption{直流和基波合成,频率为 48.12Hz,峰峰值为 20.5V}
      \end{minipage}
      
      \begin{minipage}[t]{0.48\textwidth}
      \centering
      \includegraphics[width=7cm]{7.2.png}
      \caption{直流和一,三级谐波合成,频率为 50.12Hz,峰峰值为 21.23V}
      \end{minipage}

      \begin{minipage}[t]{0.48\textwidth}
      \centering
      \includegraphics[width=7cm]{7.3.png}
      \caption{直流和一,三,五级谐波合成,频率为 50.51Hz,峰峰值为 21.48V}
      \end{minipage}
    \end{figure}

  \newpage
  \item 矩形波的分解和合成
  
    信号的分解
    \begin{figure}[htbp]
      \centering
      \begin{minipage}[t]{0.48\textwidth}
      \centering
      \includegraphics[width=6cm]{3.1.png}
      \caption{一级谐波,频率为 49.97Hz,峰峰值为 11.4V}
      \end{minipage}
      \begin{minipage}[t]{0.48\textwidth}
      \centering
      \includegraphics[width=6cm]{3.2.png}
      \caption{二级谐波,频率为 102.3Hz,峰峰值为 10.2V}
      \end{minipage}

      \begin{minipage}[t]{0.48\textwidth}
      \centering
      \includegraphics[width=6cm]{3.3.png}
      \caption{三级谐波,频率为 151.4Hz,峰峰值为 9.08V}
      \end{minipage}
      \centering
      \begin{minipage}[t]{0.48\textwidth}
      \centering
      \includegraphics[width=6cm]{3.4.png}
      \caption{四级谐波,频率为 201.2Hz,峰峰值为 7.48V}
      \end{minipage}

      \begin{minipage}[t]{0.48\textwidth}
      \centering
      \includegraphics[width=6cm]{3.5.png}
      \caption{五级谐波,频率为 245.6Hz,峰峰值为 4.06V}
      \end{minipage}
      \begin{minipage}[t]{0.48\textwidth}
      \centering
      \includegraphics[width=6cm]{3.6.png}
      \caption{六级谐波,频率为 287.9Hz,峰峰值为 3.12V}
      \end{minipage}
      \begin{minipage}[t]{0.48\textwidth}
        \centering
        \includegraphics[width=5cm]{9.png}
        \caption{直流分量,值为 5.4V}
      \end{minipage}
    \end{figure}

    \newpage
    信号的合成
    \begin{figure}[htbp]
      \centering
      \begin{minipage}[t]{0.48\textwidth}
      \centering
      \includegraphics[width=7cm]{8.1.png}
      \caption{直流和基波合成,频率为 50.01Hz,峰峰值为 20.80V}
      \end{minipage}
      \begin{minipage}[t]{0.48\textwidth}
      \centering
      \includegraphics[width=7cm]{8.2.png}
      \caption{直流,一,二级谐波合成,频率为 50.13Hz,峰峰值为 21.62V}
      \end{minipage}

      \begin{minipage}[t]{0.48\textwidth}
      \centering
      \includegraphics[width=7cm]{8.3.png}
      \caption{直流,一,二,三级谐波,频率为 49.77Hz,峰峰值为 21.24V}
      \end{minipage}
      \centering
      \begin{minipage}[t]{0.48\textwidth}
      \centering
      \includegraphics[width=7cm]{8.4.png}
      \caption{直流,一,二,三,四级谐波,频率为 46.02Hz,峰峰值为 21.52V}
      \end{minipage}

      \begin{minipage}[t]{0.48\textwidth}
      \centering
      \includegraphics[width=7cm]{8.5.png}
      \caption{直流,一,二,三,四,五级谐波,频率为 49.89Hz,峰峰值为 21.31V}
      \end{minipage}
      \begin{minipage}[t]{0.48\textwidth}
      \centering
      \includegraphics[width=7cm]{8.6.png}
      \caption{直流,一,二,三,四,五,六级谐波,频率为 50.83Hz,峰峰值为 21.57V}
      \end{minipage}
    \end{figure}


\end{enumerate}

\subsection*{七、实验总结}
\begin{enumerate}
  \item 输入信号可以分解为各级谐波信号,且各级谐波叠加起来会与输入信号越来越接近。这说明信号函数可以分解为许多谐波分量叠加的形式,与傅里叶级数展开的结果在误差范围内近似相等。
  \item 并不是所有输入信号都有 $1 \sim 6$ 级谐波,比如三角波和方波的偶次谐波峰峰值较小可近似忽略,故只有奇次谐波。
  \item 输入信号分解出的各级谐波峰峰值与傅里叶级数展开的 $A_n$ 近似相等,虽然在本次实验中,由于一些仪器接触不良以及示波器示数有些偏差,导致部分数据和理论上有一定偏差,不过大多数数据还是在误差范围内与理论值近似相等。
\end{enumerate}

\end{document}