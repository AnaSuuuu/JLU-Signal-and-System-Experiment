%!TEX program = xelatex
\documentclass[dvipsnames, svgnames,a4paper,11pt]{article}
% ----------------------------------------------------
%   吉林大学通信工程学院信号与系统实验报告
%   原作者:Huanyu Shi,2019级
%     知乎:https://www.zhihu.com/people/za-ran-zhu-fu-liu-xing
%     Github:https://github.com/huanyushi/SYSU-SPA-Labreport-Template
%   在原基础上魔改了一些,更加贴近吉林大学实验格式。
% ----------------------------------------------------

\input{Settings} % 导入模板的相关设置
\usepackage{lipsum}


%---------------------------------------------------------------------
%	正文
%---------------------------------------------------------------------

\begin{document}

\begin{table}
  \raggedleft
	\renewcommand\arraystretch{1.7}
	\begin{tabular}{|c|p{4em}|}
	\hline
	成绩 &  \\
	\hline
	教师签字 &   \\
	\hline
	\end{tabular}
\end{table}

\begin{center}
	{\kaishu \LARGE   \quad  \quad 通  \quad 信  \quad 工  \quad 程  \quad 学  \quad 院 }
  \newline
  \newline
  \newline
  \newline
  \newline
  {\kaishu \Huge 实 \quad  \quad  \quad 验  \quad  \quad  \quad 报 \quad  \quad  \quad 告}
  \newline
  \newline
  \newline
  \newline
  \newline
  {\songti \Huge  ( \quad  信  \quad 号  \quad 与  \quad 系  \quad 统 \quad)}
  \newline
  \newline
  \newline
  \newline
  \newline
  {\songti  \LARGE 实验题目:非正弦周期信号的分解与合成  \quad  \quad \quad}
\end{center}



\begin{table}[b]
	\renewcommand\arraystretch{1.7}
	\begin{tabularx}{\textwidth}{|X|X|X|X|}
	\hline
	专业:& 通信工程 &年级:& 2022级\\
	\hline
	姓名:& 苏睿杰  & 学号:& 20220826\\
	\hline
	实验时间:& 2023年11月3日 & 班级:& 42 \\
	\hline
	\end{tabularx}
\end{table}


%\clearpage
%\tableofcontents

\clearpage
\setcounter{section}{0}
\section{实验十六 \quad 非正弦周期信号的分解与合成}
\subsection*{一、实验目的}
\begin{enumerate}
  \item 用实验方法观测非正弦周期信号的分解,并与其傅利叶级数各项的频率与系数作比较。 
  \item 观测基波和其谐波的合成。
\end{enumerate}

\subsection*{二、仪器设备}
\begin{enumerate}
  \item 实验箱一台。
  \item 数字示波器。
\end{enumerate}

\subsection*{三、原理说明}
任何信号都是由各种不同频率、幅度和初相的正弦波叠加而成的。由非正弦周期信号傅里叶级数展开式可知,各次谐波为基波频率的整数倍。而第一个非周期信号包含了从零到无穷大的所有频率成分,其幅度将随谐波次数的增高而减小,直至无穷小,将被测信号加到分别调谐于其基波和各次谐波频率的电路上。从每一带通滤器的输出端可以用示波器观察到相应频率的正弦波。本实验所用的被测信号是选用 $50Hz$ 的方波、矩形波、三角波、全波和半波等。

\subsection*{四、实验内容及步骤}
\begin{enumerate}
  \item 调节函数信号发生器,使其输出为 $50Hz$ 方波。将其接至该实验模块的输入端,再细调函数信号发生器的输出,使 $50Hz$(基波)的 BPF 模块有最大的输出。然后,将各带通滤波器的输出分别接至示波器和交流亳伏表,观测各次谐波的频率和幅度,并记录之。
  \item 将方波分解所得的基波和三次谐波分量接至加法器的相应输入端,观测加法器的输出波形,并记录所得的波形。
  \item 再将方波的五次谐波分量加到加法器的相应输入端,观测相加后的波形,记录之。
  \item 分别输入 $50Hz$ 的矩形波、三角波、全波和半波信号,并分别观测谐波分量,记录波形的幅值及频率。
  \item 在加法器的输入端接入相应的各谐波分量,进行信号的合成实验,观察频率失真,并记录结果。
\end{enumerate}

\end{document}