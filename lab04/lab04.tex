%!TEX program = xelatex
\documentclass[dvipsnames, svgnames,a4paper,11pt]{article}
% ----------------------------------------------------
%   吉林大学通信工程学院信号与系统实验报告
%   原作者:Huanyu Shi,2019级
%     知乎:https://www.zhihu.com/people/za-ran-zhu-fu-liu-xing
%     Github:https://github.com/huanyushi/SYSU-SPA-Labreport-Template
%   在原基础上魔改了一些,更加贴近吉林大学实验格式。
% ----------------------------------------------------

\input{Settings} % 导入模板的相关设置
\usepackage{lipsum}


%---------------------------------------------------------------------
%	正文
%---------------------------------------------------------------------

\begin{document}

\begin{table}
  \raggedleft
	\renewcommand\arraystretch{1.7}
	\begin{tabular}{|c|p{4em}|}
	\hline
	成绩 &  \\
	\hline
	教师签字 &   \\
	\hline
	\end{tabular}
\end{table}

\begin{center}
	{\kaishu \LARGE   \quad  \quad 通  \quad 信  \quad 工  \quad 程  \quad 学  \quad 院 }
  \newline
  \newline
  \newline
  \newline
  \newline
  {\kaishu \Huge 实 \quad  \quad  \quad 验  \quad  \quad  \quad 报 \quad  \quad  \quad 告}
  \newline
  \newline
  \newline
  \newline
  \newline
  {\songti \Huge  ( \quad  信  \quad 号  \quad 与  \quad 系  \quad 统 \quad)}
  \newline
  \newline
  \newline
  \newline
  \newline
  {\songti  \LARGE 实验题目:离散信号的频谱  \quad  \quad \quad}
\end{center}



\begin{table}[b]
	\renewcommand\arraystretch{1.7}
	\begin{tabularx}{\textwidth}{|X|X|X|X|}
	\hline
	专业:& 通信工程 &年级:& 2022级\\
	\hline
	姓名:& 苏睿杰  & 学号:& 20220826\\
	\hline
	实验时间:& 2023年11月3日 & 班级:& 42 \\
	\hline
	\end{tabularx}
\end{table}


%\clearpage
%\tableofcontents

\clearpage
\setcounter{section}{0}
\section{实验十九 \quad 线性系统的频率特性}
	
\subsection*{一、实验目的}
\begin{enumerate}
	\item 观察离散信号并绘制其频谱,了解离散信号频谱的特点。
	\item 验证抽样定理。
\end{enumerate}

\subsection*{二、实验设备}
\begin{enumerate}
	\item 信号与系统试验箱。
	\item 数字信号发生器。
	\item 数字示波器。
	\item 选频电平表。
\end{enumerate}

\subsection*{三、实验原理}
\begin{enumerate}
  \item 离散时间信号可以从离散信号源获得,也可以从连续时间信号经抽样得到,抽样信号 $fs(t)$ 可以看成连续信号 $f(t)$ 和一组开关函数 $s(t)$ 的乘积,即 $fs(t)= f(t) s(t)$。
  
  $s(t)$ 是一组周期性窄脉冲,周期 $T$ 称为抽样周期。其倒数 $f_s = \dfrac{1}{Ts}$称为抽样频率。
  
  若连续信号 $f(t)$ 的频谱如图2所示,则以 $f(t)$ 取样获得的取样信号 $f_s(t)$ 的频谱包括了原连续信号 $f(t)$ 的频谱以及无限个经过平移的原信号频谱,平移的频率间隔等于取样频率 $\omega_s = \dfrac{2\pi}{T_s}$,如图3所示,如果开关函数是周期性矩形脉冲,且脉冲宽度 $\tau$ 不为零时,则取信号 $f_s(t)$ 的频谱 $F_s(\omega)$ 的包络线按 $\dfrac{\sin x}{x}$ 的规律接减($X = \dfrac{n\pi \tau}{T_s}$)。
  
  取样信号 $f_s(t)$ 的频谱与连续时间信号测试方法一样,此时须注意频谱的周期性延拓。
  
  \item 正如测得了足够的实验数据以后,我们可以在坐标纸上把一系统数据点连起来,得到一条光滑的曲线。抽样信号在一定条件下也可以恢复到原信号,只要用一截止频率等于原信号频谱中最高频率 $\omega_m$ 的低通滤波器滤除高频分量,而仅存原信号频谱的频率成分,这样,低通滤波器的输出是得到恢复的原信号。
  
  根据采样定理,原信号得以恢复的条件是取样频率 $f_s \ge 2B$。$f_s$ 为取样频率,$B$ 为原信号的有效频带宽度。当取样频率 $f_s < 2B$ 时,取样信号的频谱会发生混迭。如图4所示,此时,我们无法用低通滤波器获得原信号频谱的全部信息内容。
  
  实验中选用 $f_s<2B$,$f_s=2B$,$f_s > 2B$ 三种抽样频率对连续信号进行抽样,以验证抽样定理一要使信号抽样后能不失真地还原,抽样频率 $f_s$ 必须大于信号频谱中最高频率的两倍。

  \item 有效频带宽度:严格地说,周期性信号所包含的谐波分量有无限多,不过由于谐波振幅随频率增高而减小,通常只考虑频率较低的一些分量就够了。从 $0$ 频率到需要考虑最高次谐波频率间的频段称为信号的频带宽度。信号频带宽度的具体定义视情况而定,有时将谐波振幅下降到最大值的 $\dfrac{1}{K}$(例如 $\dfrac{1}{10}$)的频段称为信号的频带宽度。本实验对于周期正弦信号频率为 $25KHz$,从频谱图可以看出,该信号为单频信号,抽样脉冲控制信号频率 $f_s$ 取 $5OKHz$。对于周期方波和三角波信号,若其基频率为 $5KHz$,从频谱图可以看出,可取 $5$ 倍于基波频率作为有效的频带宽度,则抽样脉冲控制信号的频率 $f_s$ 取 $50KHz$。
  
\end{enumerate}

\end{document}
